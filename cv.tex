%-------------------------------------------------------------------
% Main CV document
%-------------------------------------------------------------------

\documentclass[a4paper,12pt]{article}

% Load custom style file
\usepackage{cv-style}

% Begin document
\begin{document}

% non-numbered pages
\pagestyle{empty} 

%----------------------------------------------------------------------------------------
%	TITLE
%----------------------------------------------------------------------------------------
\begin{tabularx}{\linewidth}{@{}c@{}}
\namefont{Conor Hayes} \\[7.5pt]
\href{https://github.com/cwoodhayes}{\raisebox{-0.05\height}\faGithub\ cwoodhayes} \ $|$ \ 
\href{https://linkedin.com/in/cwoodhayes}{\raisebox{-0.05\height}\faLinkedin\ cwoodhayes} \ $|$ \ 
\href{https://cwhayes.squarespace.com}{\raisebox{-0.05\height}\faGlobe\ cwhayes.squarespace.com} \ $|$ \ 
\href{mailto:cwoodhayes@gmail.com}{\raisebox{-0.05\height}\faEnvelope\ cwoodhayes@gmail.com} \ $|$ \
Chicago, IL (willing to relocate) \\
\end{tabularx}

%----------------------------------------------------------------------------------------
%	SUMMARY
%----------------------------------------------------------------------------------------
\section{Summary}
Creative engineering leader building across the full stack of robotics, from AI/ML algorithms to firmware to cloud software. Experienced in embedded systems, robotics software, and full-stack cloud infrastructure. Passionate about developing reliable, human-centered robotic systems.

%----------------------------------------------------------------------------------------
%	SKILLS
%----------------------------------------------------------------------------------------
\section{Skills}
\begin{tabularx}{\linewidth}{@{}l X@{}}
\textbf{Embedded \& Robotics} & Embedded systems, IoT, real-time firmware (ARM Cortex-M, NRF5x, Zephyr, Arduino), ROS2, Linux, computer vision \\
\textbf{Software Engineering} & Cloud and backend development on AWS (ECS, EC2, S3, RDS), CI/CD, containerization with Docker, DevOps pipelines \\
\textbf{Programming Languages} & C, Python, C++, Rust, SQL, TypeScript, Matlab, Java \\
\textbf{Tools} & Git, VS Code, Jenkins, MATLAB, OpenCV, NumPy, Pandas \\
\textbf{Leadership} & Technical architecture, team management, deep-tech product development, and interdisciplinary collaboration \\
\end{tabularx}

%----------------------------------------------------------------------------------------
%	WORK EXPERIENCE
%----------------------------------------------------------------------------------------
\section{Work Experience}

\begin{joblong}{Consultant — Conor Hayes Software Consulting}{2023 -- Present}
\item Provide technical expertise and product development guidance for deep-tech hardware/software companies.
\item Led architecture and implementation of a reusable hardware testing framework now powering two \$500k test racks for space and fusion clients.
\item Designed database schema and AWS cloud architecture for a digital-twin battery testing system in satellite assembly.
\item Delivered computer vision support for an autonomous car refueling robot.
\item For \textbf{Wesper}: Redesigned BLE streaming protocol for an NRF52-based sleep sensor (8× bandwidth increase) and migrated backend to CI/CD with containerized AWS deployment.
\item Led R\&D on IMU/EMG-based deep learning system for human body pose estimation for both academic and commercial use.
\end{joblong}

\begin{joblong}{Software Engineer — Wesper}{2020 -- 2022, New York City}
\item Led development of device firmware, backend, and cloud infrastructure for an FDA-approved “sleep lab at home” product.
\item Scaled system from 8-person prototype team to 23-person production team serving 1000+ patients monthly.
\item Tech stack: Python, C, NRF52, AWS (EC2, S3, RDS), BLE (C/JS), Matlab, Jenkins.
\end{joblong}

\begin{joblong}{Software \& Electrical Engineering Intern — Honeybee Robotics}{May -- Aug 2018, Pasadena, CA}
\item Built middleware and electronics for robotic assemblies.
\item Developed hardware drivers, GUIs, and control algorithms for an internal ROS-like testing framework.
\end{joblong}

\begin{joblong}{Flight Electronics Intern — NASA Jet Propulsion Laboratory}{Jan -- Aug 2017, Pasadena, CA}
\item Developed the hardware and software test suite for the first deep-space-capable cubesat C\&DH board (Sphinx).
\item Credited as inventor on NASA Copyright of Invention NPO 51462-CP.
\end{joblong}

%----------------------------------------------------------------------------------------
%	EDUCATION
%----------------------------------------------------------------------------------------
\section{Education}
\begin{tabularx}{\linewidth}{@{}l X@{}}
2015 -- 2019 & \textbf{University of Southern California}, B.S. in Computer Engineering \& Computer Science \hfill GPA: 3.82 (Magna Cum Laude) \\
& Minors: Chinese for the Professions, Engineering Honors, Thematic Option (Honors in Liberal Arts) \\
& Avionics Lead, USC Rocket Propulsion Lab — Co-led first undergraduate team to launch a rocket to space (Traveler IV). \\
\end{tabularx}

%----------------------------------------------------------------------------------------
%	AWARDS
%----------------------------------------------------------------------------------------
\section{Awards}
\begin{itemize}[nosep, leftmargin=1em, itemsep=3pt, label=--]
\item AIAA Achievement Award for contributions to USC’s Traveler IV rocket project.
\item National Academy of Engineering Grand Challenges Scholar (1 of 40 nationwide).
\item USC Renaissance Scholar (0.9\% of graduating class).
\item USC Trustee Scholar (4-year, half-tuition merit scholarship; one of 100 awarded).
\end{itemize}

%----------------------------------------------------------------------------------------
%	PROJECTS / MUSIC
%----------------------------------------------------------------------------------------
\section{Projects}
\begin{tabularx}{\linewidth}{@{}l X@{}}
\textbf{Independent Music Career} &  \normalsize{Released original indie rock under artist name “Wise John.” Accumulated 700,000+ Spotify streams. Sold out 170-capacity show at NYC’s Mercury Lounge (2024). Developing follow-up album in 2025.} \\
\end{tabularx}

\vfill
\center{\footnotesize Last updated: \today}

\end{document}
